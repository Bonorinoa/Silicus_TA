\documentclass[10pt]{article}
\usepackage[utf8]{inputenc}
\usepackage[T1]{fontenc}
\usepackage{graphicx}
\usepackage[export]{adjustbox}
\graphicspath{ {./images/} }
\usepackage{amsmath}
\usepackage{amsfonts}
\usepackage{amssymb}
\usepackage[version=4]{mhchem}
\usepackage{stmaryrd}
\usepackage{hyperref}
\hypersetup{colorlinks=true, linkcolor=blue, filecolor=magenta, urlcolor=cyan,}
\urlstyle{same}

\begin{document}
\section*{Short-Run Macroeconomics}
So far, we have been exploring the economic system in the long-run. The Classical view of the economy. This time-horizon allowed us to make several assumptions about economic behavior and trends. Now it's time for us to complicate things a bit further and place our focus on the short-run, where prices are sticky, thus markets may not clear, and factors of production are not fully employed due to increased uncertainty.

\section*{Key Differences: Long-Run vs. Short-Run}
\begin{center}
\begin{tabular}{ll}
\multicolumn{1}{c}{Long-Run (Classical)} & \multicolumn{1}{c}{Short-Run (Keynesian)} \\
Flexible prices & Sticky prices \\
Markets clear & Markets may not clear \\
Factors of production fully employed & Factors of production may not be fully employed \\
Classical dichotomy holds & Classical dichotomy breaks down \\
\end{tabular}
\end{center}

The main reason why prices are sticky, especially wages, is due to contracts which are signed for a predetermined period of time. Thus the rate set in the contract fixes that price for a while. Since they are fixed, they cannot be flexible, and thus the market may not clear at any given time.\\
Another cause is called "menu costs", which refers to the unwillingness of businesses to constantly (or dynamically) change their prices to reflect current economic conditions. Restaurants don't want to print new menus often and stores don't want to change price tags every other day. Although, this is something that may change in the near future with the increasing digitization of the economy. For example, grocery shops in nordic countries have digital price tags that update according to an algorithm; a phenomena called dynamic pricing.

How might widespread adoption of dynamic pricing technology affect short-run economic dynamics? What are potential benefits and drawbacks for consumers and businesses?

Can you think of another scenario in which there is a strong incentive to regularly update menus or price tags? High inflation!

Recall that in the Classical world, money supply affects nominal variables (those unadjusted and that represent money goods); but does not cause fluctuations in real variables (those adjusted for changes in price and that represent physical goods) like real GDP or unemployment because prices have time to adjust. We defined this as the Classical Dichotomy. In the long-run, changes in nominal variables do not cause changes in real variables.

In the short run, however, many prices do not respond to changes in monetary policy. A reduction in the money supply does not immediately cause all firms to cut the wages they pay, all stores to change the price tags on their goods, and all restaurants to print new menus. Instead, there is little immediate change in many prices; that is, many prices are sticky. This short-run price stickiness implies that the short-run impact of a change in the money supply is not the same as the long-run impact (Mankiw, Chapter 10).

Because of this "lagged" reaction to changes in the money supply, we expect nominal variables to in fact influence real variables. The Classical Dichotomy no longer holds. Real variables will need to adjust to keep the economic system close to equilibrium.

\section*{Business Cycles}
Moreover, because labor is not fully and efficiently employed, new phenomena arise such as recessions and booms (recall that we could not have recessions in the classical world since labor is fully employed). This variability is defined as business cycles.\\
An important distinction between these time-horizons is that the long-run trend leading to higher standards of living from generation to generation is not associated with any long-run trend in the rate of unemployment. This is a consequence of what assumption of the classical model? By contrast, short-run movements in GDP are strongly correlated with the utilization of the economy's labor force. The declines in the production of goods and services that occur during recessions are always associated with increases in joblessness (Mankiw, Chapter 10).

\section*{Okun's Law}
This negative relationship between GDP and Unemployment was first observed empirically by Arthur Okun. Thus the name Okun's Law.\\
\includegraphics[max width=\textwidth, center]{2025_01_09_70c009e2a080be6fb020g-2}\\
\includegraphics[max width=\textwidth, center]{2025_01_09_70c009e2a080be6fb020g-3}

As you might intuit, businesses and government officials care much more about what will happen next year or next month rather than in the upcoming decades. Most economics jobs are in fact related to forecasting business cycles, market sentiment, commodity prices, or other short-run variables. Long-run analysis is left for academics pondering theoretical insights to guide economic policy for future generations. The main reason is that economic fluctuations directly affect politicians in government and businesses trying to maximize profit in the near future. Government officials use these forecasts to design policies. Economic forecasts are, therefore, an input into policy planning.

Can you think of factors that might cause deviations from Okun's Law? How might technological advancements or changes in labor market structure affect this relationship?

\section*{Lagging vs Leading Indicators}
There are two important types of short-run economic variables.

\begin{itemize}
  \item Lagging indicators: Variables that fluctuate after a policy has been enacted, or after some event occurred. They confirm long-term trends, but they do not predict them.\\
Examples of lagging indicators include the unemployment rate, corporate profits, and labor cost per unit of output. These indicators are useful for identifying the strength of long-term trends but are less effective for predicting changes in the business cycle.
  \item Leading indicators: Variables that often fluctuate before we see movements in the overall economy. They are used to predict future economic activity. Examples of leading indicators include stock market returns, manufacturing activity, and the number of new businesses incorporated. These indicators can provide early warnings of changes in the economic cycle and help policymakers and businesses make informed decisions.
\end{itemize}

Each month the Conference Board, a private economics research group, announces the index of leading economic indicators. This index includes ten data series that are often used to forecast changes in economic activity about six to nine months into the future. Here is the latest composition of the Leading Economic Index

\begin{enumerate}
  \item Average weekly hours in manufacturing
\end{enumerate}

\begin{itemize}
  \item Data Source: U.S. Bureau of Labor Statistics (BLS)
  \item URL: BLS Average Weekly Hours
\end{itemize}

\section*{2. Average weekly initial claims for unemployment insurance}
\begin{itemize}
  \item Data Source: U.S. Department of Labor
  \item URL: DOL Unemployment Insurance Weekly Claims
\end{itemize}

\section*{3. Manufacturers' new orders for consumer goods and materials}
\begin{itemize}
  \item Data Source: U.S. Census Bureau
  \item URL: Census Bureau New Orders
\end{itemize}

\begin{enumerate}
  \setcounter{enumi}{3}
  \item ISM® Index of New Orders
\end{enumerate}

\begin{itemize}
  \item Data Source: Institute for Supply Management (ISM)
  \item URL: ISM Report on Business
\end{itemize}

\section*{5. Manufacturers' new orders for nondefense capital goods excluding aircraft orders}
\begin{itemize}
  \item Data Source: U.S. Census Bureau
  \item URL: Census Bureau New Orders
\end{itemize}

\section*{6. Building permits for new private housing units}
\begin{itemize}
  \item Data Source: U.S. Census Bureau
  \item URL: Census Bureau Building Permits
\end{itemize}

\section*{7. S\&P 500® Index of Stock Prices}
\begin{itemize}
  \item Data Source: S\&P Dow Jones Indices
  \item URL: S\&P 500 Index
\end{itemize}

\begin{enumerate}
  \setcounter{enumi}{7}
  \item Leading Credit Index ${ }^{\text {TM }}$
\end{enumerate}

\begin{itemize}
  \item Data Source: The Conference Board
  \item URL: Conference Board Leading Credit Index
\end{itemize}

\begin{enumerate}
  \setcounter{enumi}{8}
  \item Interest rate spread (10-year Treasury bonds less federal funds rate)
\end{enumerate}

\begin{itemize}
  \item Data Source: Federal Reserve Bank of St. Louis (FRED)
  \item URL: FRED Interest Rate Spread
\end{itemize}

\begin{enumerate}
  \setcounter{enumi}{9}
  \item Average consumer expectations for business conditions
\end{enumerate}

\begin{itemize}
  \item Data Source: University of Michigan: Surveys of Consumers
  \item URL: University of Michigan Consumer Sentiment\\
\includegraphics[max width=\textwidth, center]{2025_01_09_70c009e2a080be6fb020g-5}
\end{itemize}

Additional Reading

\begin{itemize}
  \item US Leading Indicators (\href{http://conference-board.org}{conference-board.org})
\end{itemize}

\end{document}