\documentclass[10pt]{article}
\usepackage[utf8]{inputenc}
\usepackage[T1]{fontenc}
\usepackage{graphicx}
\usepackage[export]{adjustbox}
\graphicspath{ {./images/} }
\usepackage{amsmath}
\usepackage{amsfonts}
\usepackage{amssymb}
\usepackage[version=4]{mhchem}
\usepackage{stmaryrd}

\begin{document}
\title{Methodology, Questions, and Data in Macroeconomics}

\section*{Scientific Method}
The scientific method is a cornerstone of modern scientific inquiry. It offers a structured framework to explore phenomena, test theories, and derive conclusions based on empirical evidence. In economics, applying the scientific method ensures that conclusions about economic behavior and policy are based on systematic and unbiased investigations.

\subsection*{Steps of the Scientific Method}
\begin{enumerate}
  \item[1.] \textbf{Observation}: This step involves gathering data and noticing phenomena that prompt questions. For instance, an economist might observe a sudden increase in unemployment rates.

  \item[2.] \textbf{Formulating a Question}: Based on observations, economists formulate questions to narrow down the scope of their inquiry. For example, "What factors are contributing to the rise in unemployment?"
  
  \item[3.] \textbf{Hypothesis Development}: A hypothesis is a tentative answer to the formulated question. It should be specific and testable. For instance, "An increase in automation is leading to higher unemployment rates."
  
  \item[4.] \textbf{Experimentation}: This step involves designing experiments or models to test the hypothesis. In economics, this could involve econometric models, natural experiments, or controlled experiments. For example, using regression analysis to test the relationship between automation levels and unemployment rates or identifying an exogenous event like a policy that directly impacts the level of automation.
  
  \item[5.] \textbf{Analysis}: After experimentation, data is analyzed to determine whether the results support or refute the hypothesis.
  
  \item[6.] \textbf{Conclusion}: Based on the analysis, economists draw conclusions about the hypothesis. If the hypothesis is supported, it may contribute to theory development. If not, it may be revised or rejected, prompting further investigation.
\end{enumerate}

\includegraphics[max width=\textwidth, center]{2025_01_09_f30ef864d5819b818b0dg-1}

\subsection*{Importance in Economics}
Applying the scientific method in economics is crucial for several reasons:

\begin{itemize}
  \item \textbf{Objectivity}: It minimizes biases, ensuring that conclusions are based on empirical evidence rather than subjective opinions.
  
  \item \textbf{Reproducibility}: Other researchers can replicate studies to verify results, adding robustness to economic theories.
  
  \item \textbf{Systematic Investigation}: It provides a structured approach to exploring economic questions, making the research process more organized and thorough.
\end{itemize}

\subsubsection*{Example in Macroeconomics}
Consider the question: "Do fiscal stimulus measures reduce unemployment during a recession?" An economist might hypothesize that increased government spending leads to job creation. To test this, they could collect data on unemployment and government spending during past recessions, use econometric models to analyze the relationship, and draw conclusions based on the statistical evidence.

\section*{Macroeconomic Methodology}
Macroeconomic methodology encompasses the various approaches and techniques that economists use to understand, analyze, and predict macroeconomic phenomena such as inflation, unemployment, and economic growth. This involves a combination of theoretical frameworks, empirical data analysis, and policy evaluation.

\subsection*{Main Approaches in Macroeconomic Methodology}
\begin{enumerate}
  \item Theoretical Models:

    \begin{itemize}
      \item Purpose: To provide a simplified representation of the economy that helps explain and predict economic behavior.
      \item Types:
      \begin{itemize}
        \item Classical Models: Focus on long-term economic growth and the neutrality of money.
        \item Keynesian Models: Emphasize short-term fluctuations and the role of government intervention.
        \item New Keynesian Models: Integrate microeconomic foundations into Keynesian economics, incorporating aspects like price stickiness and market imperfections.
      \end{itemize}
    \end{itemize}

  \item Empirical Analysis:

    \begin{itemize}
      \item Purpose: To validate theoretical models using real-world data and statistical techniques.
      \item Methods:
      \begin{itemize}
        \item Econometrics: Applying statistical methods to economic data to estimate relationships and test hypotheses.
        \item Time-Series Analysis: Examining data points collected or recorded at specific time intervals to identify trends, cycles, and seasonal variations.
      \end{itemize}
    \end{itemize}

  \item Policy Evaluation:

    \begin{itemize}
      \item Purpose: To assess the impact of economic policies on macroeconomic variables.
      \item Methods:
      \begin{itemize}
        \item Counterfactual Analysis: Estimating what would have happened in the absence of a particular policy.
        \item Cost-Benefit Analysis: Weighing the total expected costs against the benefits of a policy.
      \end{itemize}
    \end{itemize}

\end{enumerate}

\subsection*{Role of Assumptions in Macroeconomic Models}
Assumptions are simplifications that make models tractable and allow us to focus on essential elements of the economy. Common assumptions include rational behavior, market equilibrium, and perfect information. However, the realism and applicability of these assumptions are critical for the model's usefulness and applicability. You will always have to make assumptions, it doesn't matter if you are building an economic model, forecasting a budget, or simulating price trajectories of a stock. So, a huge part of your efforts when modeling or studying a new model must be placed on understanding the underlying assumptions. The necessity of assumptions stem from limitations inherent to the tools and instruments at our disposal.

\subsection*{Challenges in Macroeconomic Methodology}
\begin{enumerate}
  \item Data Limitations: Reliable and high-frequency macroeconomic data can be scarce, especially for developing countries. You can impute missing data, but this requires assuming certain properties of the data distribution.

  \item Model Specification: Choosing the correct model form and variables is crucial but challenging. You will always be missing some important variable, this is called Omitted Variable Bias (OVB) in Econometrics, but maybe you can find workarounds to explain your results with reasonable assumptions.

  \item Complexity of Economic Systems: Economies are influenced by numerous interrelated factors, making it difficult to isolate individual effects. In Econometrics, we call this Heterogeneity bias and assumptions about the variables of interest are needed to derive causal relationships from your model.
\end{enumerate}

-----------------------------------------
\subsubsection*{"How do you think the increasing availability of big data and advanced computing power might change macroeconomic modeling in the future?"}
More granular and high-frequency data which can be processed by ML algorithms to uncover patterns otherwise hidden by lack of data or simpler statistical models. This can help improve the prediction accuracy of our models, although often at the cost of interpretability. Also, the higher availability of data might help paint a more realistic picture of the economic system, helping design more targeted and effective policy interventions.

There is a big debate nowadays about how to incorporate ML techniques into econometrics. As I mentioned on Tuesday, we care about explaining events not so much predicting them. Albeit this is mainly because our current tools cannot predict much ex ante. This has been the subject of much criticism of the field of economics, but it might soon change; complexity tries to provide alternative frameworks like simulations to do so. Check out Susan Athey, Alberto Abadie, Brian Arthur, Doyne Farmer, and Guido Imbens.
-----------------------------------------

\section*{-}
\section*{Additional Reading}
\begin{enumerate}
  \item Impact of ML in Economics - Athey
  \item ML models every economist should know - Athey and Imbens
  \item Statistical modeling: Two cultures - Leo Breiman
\end{enumerate}

\section*{Fundamental Questions, Unsolved Problems, and New Perspectives on Macro Theories}

\subsection*{Fundamental Questions in Macroeconomics}
\begin{enumerate}
  \item What Determines Economic Growth?

    \begin{itemize}
      \item Long-Run Growth: Investigating factors such as capital accumulation, technological progress, and human capital.
      \item Short-Run Fluctuations: Understanding business cycles and the role of demand-side factors.
    \end{itemize}

  \item What Causes Economic Fluctuations?

    \begin{itemize}
      \item Business Cycles: Identifying the causes of periodic expansions and contractions in economic activity.
      \item External Shocks: Assessing the impact of events like oil price changes, financial crises, and geopolitical tensions.
    \end{itemize}

  \item How Do Monetary and Fiscal Policies Affect the Economy?

    \begin{itemize}
      \item Monetary Policy: Exploring the influence of central bank actions on inflation, interest rates, and economic output.
      \item Fiscal Policy: Evaluating the effects of government spending and taxation on aggregate demand and economic stability.
    \end{itemize}

  \item What Determines Inflation?

    \begin{itemize}
      \item Inflation Dynamics: Analyzing the relationship between money supply, demand, and price levels.
      \item Expectations: Understanding how inflation expectations shape actual inflation outcomes.
    \end{itemize}

  \item What Drives Unemployment?

    \begin{itemize}
      \item Labor Market Dynamics: Examining factors like labor supply and demand, wage setting, and structural changes.
      \item Policy Interventions: Assessing the effectiveness of policies aimed at reducing unemployment.
    \end{itemize}

\subsection*{Unsolved Problems in Macroeconomics}
\begin{enumerate}
  \item Predicting Economic Crises: Despite advances, predicting financial crises and severe recessions remains a major challenge.

    \begin{itemize}
      \item Complexity and Interconnectivity: The global economy's complexity makes it difficult to anticipate systemic risks.
    \end{itemize}

  \item Understanding Inequality: The rise in income and wealth inequality has significant macroeconomic implications.

    \begin{itemize}
      \item Impact on Growth: Investigating how inequality affects economic growth, consumption, and investment.
    \end{itemize}

  \item Policy Effectiveness: The debate over the effectiveness of monetary versus fiscal policy is ongoing.

    \begin{itemize}
      \item Interaction Effects: Understanding how these policies interact and influence each other.
    \end{itemize}

  \item Globalization: The effects of increased global economic integration on domestic economies.

    \begin{itemize}
      \item Trade and Capital Flows: Assessing the impact of international trade, capital mobility, and multinational corporations.
    \end{itemize}

\end{enumerate}

\subsection*{New Perspectives on Macro Theories}
\begin{enumerate}
  \item Behavioral Macroeconomics: Integrating insights from psychology to better understand economic decision-making.

    \begin{itemize}
      \item Rationality Assumptions: Challenging the notion of fully rational agents and exploring bounded rationality.
    \end{itemize}

  \item Agent-Based Models: Simulating interactions of individual agents to study macroeconomic phenomena.

    \begin{itemize}
      \item Complex Systems: Viewing the economy as a complex, adaptive system with emergent properties.
    \end{itemize}

  \item Digital Economy: Understanding the macroeconomic implications of digital technologies and innovation.

    \begin{itemize}
      \item Productivity and Employment: Analyzing the effects of digital transformation on productivity, labor markets, and economic structures.
    \end{itemize}
\end{enumerate}

\section*{Types of Data}
\subsection*{Endogenous vs. Exogenous Variables}
\begin{enumerate}
  \item Endogenous Variables:

    \begin{itemize}
      \item Definition: Variables determined within the context of an economic model.
      \item Examples: In a supply and demand model, the equilibrium price and quantity are endogenous variables.
      \item Importance: Understanding how these variables interact within the model helps to predict economic outcomes.
    \end{itemize}

  \item Exogenous Variables:

    \begin{itemize}
      \item Definition: Variables determined outside the economic model and imposed on it.
      \item Examples: Technological changes, government policies, and external shocks (like natural disasters).
      \item Importance: Exogenous variables influence the endogenous variables but are not explained by the model itself.
    \end{itemize}
\end{enumerate}

-----------------------------------------
\subsubsection*{"In the context of climate change and economic policy, which variables would you consider endogenous and which exogenous? Why?"}
Endogenous variables might include:

\begin{itemize}
  \item Carbon emissions from economic activities
  \item Investment in green technologies
  \item Energy consumption patterns
  \item Adaptation costs for businesses and households
\end{itemize}

These are considered endogenous because they are influenced by economic decisions and policies within the system being modeled.

Exogenous variables might include:

\begin{itemize}
  \item Global temperature rise
  \item Extreme weather events
  \item International climate agreements
  \item Technological breakthroughs in clean energy
\end{itemize}

These are typically considered exogenous because they are determined by factors outside the immediate economic system or are influenced by global factors beyond the control of individual economic actors or national policies.
-----------------------------------------

\subsection*{Stock vs. Flow Variables}

\begin{enumerate}
  \item \textbf{Stock Variables}:
    \begin{itemize}
      \item Definition: Variables measured at a specific point in time.
      \item Examples: Capital stock, national debt, money supply.
      \item Importance: Provides a snapshot of the economic condition at a given moment.
    \end{itemize}

  \item \textbf{Flow Variables}:
    \begin{itemize}
      \item Definition: Variables measured over a period of time.
      \item Examples: GDP, income, investment, government spending.
      \item Importance: Indicates economic activity and changes over time.
    \end{itemize}

\subsection*{Nominal vs. Real Variables}

\begin{enumerate}
  \item Nominal Variables:

    \begin{itemize}
      \item Definition: Variables measured in current prices, without adjusting for inflation.
      \item Examples: Nominal GDP, nominal wages.
      \item Importance: Reflects the actual monetary value at the time of measurement but can be misleading over time due to inflation.
    \end{itemize}

  \item Real Variables:

    \begin{itemize}
      \item Definition: Variables adjusted for changes in the price level, reflecting true purchasing power.
      \item Examples: Real GDP, real wages.
      \item Importance: Provides a more accurate measure of economic performance and living standards over time.
\end{itemize}

\begin{center}
  \textit{** Consider these in your essay when you describe the variables or measurements chosen **}
\end{center}

\subsection*{Data Structures}
\begin{enumerate}
  \item \textbf{Time-series}: Observations over time (seconds, days, months, years, ...)
  \item \textbf{Cross-sectional}: Observations over units (patients, countries, ...)
  \item \textbf{Panel (or longitudinal)}: Observations over time and units (year and country, day and patient, ...)
\end{enumerate}

\subsection*{Data Transformations}
\begin{enumerate}
  \item Index Numbers:

    \begin{itemize}
      \item Definition: A statistical measure to represent the relative change in a variable over time from a base period (year/month/day/hour).
      \item Metrics: Consumer Price Index (CPI), Producer Price Index (PPI).
      \item Usage: Commonly used to measure inflation and compare economic activity across different time periods.
      \item Methods:
        \begin{itemize}
          \item Simple Price Index: $P_t = (P_t / P_0) \cdot 100$
          
          \begin{itemize}
            \item Example: If a basket of goods cost $ \$100 $ in 2020 (base year) and $ \$110 $ in 2021: 2021 Index $=(\$ 110 / \$ 100) \cdot 100 = 110$
            \item Prices increased by $ 10\% $ from 2020 to 2021.
          \end{itemize}

          \item el otro index

          \begin{itemize}
            \item Example: 
            \item interpretation
          \end{itemize}

        \end{itemize}
    \end{itemize}

  \item Seasonal Adjustment:
    \begin{itemize}
      \item Definition: A technique to remove the effects of seasonal variations in data.
      \item Metrics: Seasonally Adjusted unemployment rates or sales.
      \item Usage: Helps to identify underlying trends and cycles in economic data.
      \item Methods:
        \begin{itemize}
          \item Ratio-to-Moving Average Method: $ \text{Seasonal Factor} = \frac{\text{Original Value}}{\text{Moving Average}} $
            \begin{itemize}
              \item Example:  If July sales are $\$ 12,000$ and the 12-month moving average is $\$ 10,000$ : July Seasonal Factor $=\$ 12,000 / \$ 10,000=1.2$
              \item July sales are typically $20 \%$ above the annual average.
            \end{itemize}

          \item X-13ARIMA-SEATS (more advanced but used by US Bureaus)
        \end{itemize}
    \end{itemize}

  \item Growth Rates:
    \begin{itemize}
      \item Definition: The rate at which a variable changes over a specific period (YoY, MoM, etc).
      \item Metrics: GDP growth rate, inflation rate.
      \item Usage: Provides insight into the speed and direction of economic changes.
      \item Methods:
      \begin{itemize}
        \item Simple Growth Rate $=\left(\mathrm{Y} \_\mathrm{t}-\mathrm{Y}_{-}(\mathrm{t}-1)\right) / \mathrm{Y}_{-}(\mathrm{t}-1)=\left(\mathrm{Y}_{-} \mathrm{T} / \mathrm{Y}_{-}(\mathrm{t}-1)\right)-1$
          \begin{itemize}
            \item ( $\mathrm{t}-1$ ) is your lag and depends on the data frequency.
            \item If you annual growth with monthly data then (Y\_t - Y\_(t-12)) / Y\_(t-12)
            \item Another way of accounting for seasonality

            \item Example: If GDP was $\$ 1,000$ billion in 2020 and $\$ 1,040$ billion in 2021: Growth Rate $=(\$ 1,040-\$ 1,000) / \$ 1,000=0.04$ or $4 \%$
            \item The economy grew by 4\% from 2020 to 2021.
          \end{itemize}

        \item Compound Annual Growth Rate (CAGR) $= P /cdot (1 - \frac{r}{100 \cdot t})^t $ (REVISE)
          \begin{itemize}
            \item Example:
            \item interpretation
          \end{itemize}
      \end{itemize}
      
      
    \end{itemize}


  \item Logarithmic Transformations:
    \begin{itemize}
      \item Definition: Applying logarithms to economic data to stabilize variance and linearize relationships.
      \item Metrics: Logarithm of GDP.
      \item Usage: Useful in regression analysis and for interpreting percentage changes.
      \item Methods:
      \begin{itemize}
        \item Natural Log Transformation: $\ln (\mathrm{Y})$

        \item Log Difference (for growth rates): $\Delta \ln (\mathrm{Y}) \approx\left(\mathrm{Y} \_\mathrm{t}-\mathrm{Y} \_(\mathrm{t}-1)\right) / \mathrm{Y} \_(\mathrm{t}-1)$
          \begin{itemize}
            \item Example: We can estimate a growth rate via log differences $\ln (1040)-\ln (1000)=6.947-6.908=0.039 \approx 4 \%$
            \item Try with the simple growth formula now
          \end{itemize}
      \end{itemize}
    \end{itemize}
\end{enumerate}

The log transformation can be used in regression analysis or to compare percentage changes across different scales. A regression where the independent and dependent variables are log transformed results in elasticities, very common in economics, especially labor economics.

\end{document}