\documentclass[10pt]{article}
\usepackage[utf8]{inputenc}
\usepackage[T1]{fontenc}
\usepackage{amsmath}
\usepackage{amsfonts}
\usepackage{amssymb}
\usepackage[version=4]{mhchem}
\usepackage{stmaryrd}
\usepackage{graphicx}
\usepackage[export]{adjustbox}
\graphicspath{ {./images/} }

\begin{document}
\section*{Theory of Liquidity Preference}
Keyne's argued that the interest rate reached equilibrium levels by adjusting to balance the supply and demand for money. Money, as far as we are concerned, is just one more economic asset; one that is highly liquid (easy to exchange). Since this is a theory of money supply, we start by setting up our expression for real money balances $\left(\frac{M}{P}\right)^{s}=\frac{\bar{M}}{\bar{P}}$, where M is money supply and $P$ is the price level. Keynes assumed both of these variables to be exogenous in his theory of liquidity preference. The intuition is that $M$ is set by the central bank, and $P$ is fixed in the short-run (remember the assumption of sticky prices and wages). Hence, the supply of real money balances in this simple model is independent of the interest rate (except, maybe, at the time of setting a new $M$ or renewing wage contracts. But this scenario is not considered in the short-run).

So, the supply of money is fixed thus yielding a vertical curve if plotted against the interest. In turn, the demand for money does directly depend on the interest rate. Can anyone think of why this might the case? What is the direction of this relationship (pos or neg)? Well, the interest rate is the opportunity cost of holding cash versus interest-bearing bank deposits or bonds. The higher the interest rate, the higher the opportunity cost of holding cash. Thus, the demand for money is negatively related to the interest rate. We write this expression as $\left(\frac{M}{P}\right)^{d}=L(r)$, where $L()$ is the demand function for money which depends on the interest rate. As usual, the equilibrium interest rate is that which balances supply and demand.\\
\includegraphics[max width=\textwidth, center]{2025_01_09_7593560eaab777c722b7g-1}

Mankiw, Macroeconomics, 10e, © 2019 Worth Publishers

In contrast with the long-run scenario, this interest rate is the one set by banks or bond issuers rather than the central bank. Although both are faces of the same coin. The difference is important to understand Keynes' theory because he assumes the interest rate adjusts to balance supply and demand for money, yet the fed's funds rate is exogenously determined by\\
the Federal reserve. With this caveat in mind, the logic follows by arguing that as the interest rate increases, demand for bonds or interest-bearing assets increase, thus lowering the demand for cash. But bond issuers are also in the business of making money, so the incentive now becomes to lower the interest rate and promote liquidation. So the increase in demand for bonds puts a downward pressure on the interest rate, and therefore demand for bonds drops until reaching the equilibrium point.

Translated to changes in the supply of real money balances, the Theory of Liquidity Preferences predicts that a decrease in money supply will raise interest rate and an increase will lower it because the price level is exogenous.

\section*{LM Curve}
While the IS curve plots the relationship between interest rate and income in the goods market, the LM curve describes the same relationship but in the money market. This is the last component of the IS-LM curve, together they help explain plausible shifts of AD in the short-run.

At this point, you should be making a connection with the topics covered during the first few weeks of the semester. We started by building a classical model that helped explain certain economic phenomena in the goods and loanable funds markets, under the assumptions of long-run economics. Now, we are doing the same thing but under the -more flexibleassumptions of short-run economics. The IS curve explains phenomena of interest in the goods market, and the LM curve aims to explain it for the money market. The interest rate is still the only endogenous variable in our model, because it is policymaker's main economic policy tool. The main difference, as noted at the beginning of the lectures on Short-Run Macroeconomics, is that our focus is now on demand rather than supply or production.

The only ingredient missing is Income. How will changes in income influence the market for money? How will the interest rate adjust given a change in income? Think back to our earlier discussions of the loanable funds market. Suppose an increase in income, what do you expect the effect on consumption habits to be? The more money people have, the more likely they are to increase their expenditure levels. For this they will need money. Thus we can expect the demand for real money balances to increase. Let's represent this logic in functional form\\
$\left(\frac{M}{P}\right)^{d}=L(r, Y)$.

If you are not comfortable with math notation, this relationship might seem weird. We are defining a function based on two variables that are not in the left-hand side expression. In this form, it allows us to make claims about the effects of changes in $r$ and $Y$ for a given supply of real money balances. Remember that $M$ and $P$ are exogenous in our model. So we assume, or compute, some level of supply and then reason about the effects of changes in $r$ or $Y$ on the given supply of money.

In summary, we now know that the demand for money is negatively related to interest rates but positively related to income. Therefore, according to Keynes' theory, an increase in income leads to an increase in interest rates. Both inputs increase and hence we get a shift outward in the demand for money ( $\mathrm{L}(\mathrm{r}, \mathrm{Y}$ ). The LM curve represents this relationship between income and interest rate.\\
(a) The Market for Real Money Balances\\
\includegraphics[max width=\textwidth, center]{2025_01_09_7593560eaab777c722b7g-3(1)}\\
(b) The LM Curve\\
\includegraphics[max width=\textwidth, center]{2025_01_09_7593560eaab777c722b7g-3}

Mankiw, Macroeconomics, 10e, © 2019 Worth Publishers

Each point in the LM curve represents an equilibrium in the money market, and the curve shows how the equilibrium interest rate depends on income. The higher the level of income, the higher the demand for real money balances, and the higher the equilibrium interest rate. For this reason, the LM curve slopes upward..

\section*{Effect of Monetary Policy}
We draw the LM curve for a given level of money demand, so if M/P changes then LM curve shifts. Suppose the Fed decreases money supply from MS1 to MS2, hence money demand falls from MS1/P to MS2/P. In turn, this raises the interest rate and shifts the LM curve upwards.\\
\includegraphics[max width=\textwidth, center]{2025_01_09_7593560eaab777c722b7g-3(2)}

In summary, the LM curve shows the combinations of the interest rate and income that are consistent with equilibrium in the market for real money balances. The LM curve is drawn for a given supply of real money balances. Decreases in the supply of real money balances shift the LM curve upward. Increases in the supply of real money balances shift the LM curve downward.

\section*{Putting Everything Together: IS-LM model}
Combining both of these tools, we can now develop our first model of short-run equilibrium. Recall the IS and LM equations\\
$I S:=Y=C(Y-\bar{T})+I(r)+\bar{G}$\\
$L M:=\frac{\bar{M}}{\bar{P}}=L(r, Y)$

The IS curve provides the combinations of $r$ and $Y$ that satisfy the equation representing the goods market, and the LM curve provides the combinations of $r$ and $Y$ that satisfy the equation representing the money market. In conjunction, they help us find a theoretical equilibrium level of interest rate and income.\\
\includegraphics[max width=\textwidth, center]{2025_01_09_7593560eaab777c722b7g-4}

Mankiw, Macroeconomics, 10e, © 2019 Worth Publishers

The equilibrium of the economy is the point at which the IS curve and the LM curve cross. This point gives the interest rate $r$ and income $Y$ that satisfy conditions for equilibrium in both the goods market and the money market. In other words, at this intersection, actual expenditure equals planned expenditure, and the demand for real money balances equals the supply.


\end{document}