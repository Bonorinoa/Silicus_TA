\documentclass[10pt]{article}
\usepackage[utf8]{inputenc}
\usepackage[T1]{fontenc}
\usepackage{amsmath}
\usepackage{amsfonts}
\usepackage{amssymb}
\usepackage[version=4]{mhchem}
\usepackage{stmaryrd}
\usepackage{graphicx}
\usepackage[export]{adjustbox}
\graphicspath{ {./images/} }

\begin{document}
\section*{Classical Demand}
In the last lecture, we reviewed the supply side of the classical model. The factors of production K and L , the production capability determined by the production function, the properties of the Cobb-Douglas production function, calculating the optimal number of inputs to maximize profits, and how to compute the distribution of income amongst the factors of production. The key takeaway was that firms, under DRS constraints, will produce until the marginal products equal marginal costs of the factors of production. That is, the cost of an additional worker (real wage) equals the marginal revenue generated by the extra worker; or the cost of an additional unit of capital (real rental cost) equals the marginal revenue generated by the extra capital.

But what determines demand for such products? According to the national income accounts identity, GDP is composed of four key components: Consumption, Investment, Government purchases, and Net exports. Assuming a closed economy (no imports or exports), we get that $G D P=C+\boldsymbol{+} \boldsymbol{G}$.

Households consume some of the economy's output, firms and households use some of the output for investment, and the government buys some of the output for public purposes. We want to see how GDP is allocated among these three uses.

\section*{Consumption}
Let's start with consumption, the biggest contributor to GDP. Households receive income from their labor and their ownership of capital, pay taxes to the government, and then decide how much of their after-tax income to consume and how much to save. The income that all households receive (in the aggregate) equals the total amount of output produced ( $\mathbf{Y}$ ). Hence, GDP = $\mathrm{Y}=$ Total Income.

The government then taxes households' income, say a lump sum amount $\mathbf{T}$, leaving them with disposable (after-tax) income Y-T. In economics, we assume (although this is well grounded in empirical observation and theory) that consumption is primarily determined by disposable income. It is intuitive, after paying taxes and debt you will allocate the remaining income to either consumption or savings. Thus, we define the consumption function as a function of disposable income\\
$C=C(Y-T)$

\section*{Where:}
\begin{itemize}
  \item C: Consumption
  \item Y: Total income (equivalent to output in the classical model)
  \item T: Taxes (assumed to be lump-sum in this simple model)\\
\includegraphics[max width=\textwidth, center]{2025_01_09_19fdeaed3c051bee82b7g-2}
\end{itemize}

Mankiw, Macroeconomics, 10e, © 2019 Worth Publishers

The slope of the consumption function tells us how much consumption increases when disposable income increases by one dollar, defined as the marginal propensity to consume (MPC).\\
$M P C=\frac{\partial C}{\partial(Y-T)} \approx \frac{\Delta C}{\Delta(Y-T)}$

For example, if the MPC is 0.8 , it means that for every additional dollar of disposable income, 80 cents will be spent on consumption. Here is a simple numerical example:

\begin{center}
\begin{tabular}{|l|l|}
\hline
Disposable Income & Consumption \\
\hline
100 & 160 \\
\hline
180 & 210 \\
\hline
\end{tabular}
\end{center}

Then, MPC $\approx \frac{\Delta C}{\Delta(Y-T)}=\frac{50}{80}=0.625$

The MPC is classically assumed to be stable and between 0 and 1 , but there is recent evidence that suggests this need not be the case. If you borrowed more money than what you earned, then it will be greater than 1 . Conversely, if more income discourages consumption then it will be less than 0 (e.g., if there is a higher incentive to save the additional income rather than spend it).

How might changes in expectations about future income affect the consumption function? Can you think of real-world scenarios where the MPC might change significantly?

Consider the cash transfers from COVID-19. We would expect that if people had more money, they would spend more. Therefore, helping drive growth in the economy. But this is not really what we observed after the fact.

First, many households increased their savings due to uncertainty about the future of the economy and their job. This translated to a lower than expected MPC. But, of course, this effect was different amongst different income groups. Lower-income groups have liquidity constraints, so they spend a larger portion of the stimulus (i.e, higher MPC). Higher income groups saved a larger proportion or speculated in the stock market since it was basically "free money" for them. We got the gamestop frenzy because of that. This is Heterogeneity bias at its best.

Second, there wasn't much to spend on! Many stores were closed, people were getting sick (a huge cost), and many services were not available for a while.

Third, households with outstanding debt used the stimulus to repay part of these debts. Since this is not accounted as consumption, we observe a lower MPC than expected.

Limitations of the Classical Consumption Function:

\begin{enumerate}
  \item It assumes that current disposable income is the primary determinant of consumption, ignoring factors like wealth, credit availability, and expectations.
  \item It doesn't account for differences in consumption patterns across income levels or changes in income distribution.
  \item The assumption of a stable MPC may not hold during economic crises or rapid growth periods.
\end{enumerate}

\section*{Investment}
We now proceed to the second component: Investment. Both firms and households purchase investment goods. Firms buy investment goods to add to their stock of capital and to replace existing capital as it wears out. Households buy new houses, which are also part of investment. Of course, how much income to allocate to investment goods will be conditioned on how likely these are to be profitable (returns exceed the costs).

The main determinant of profitability of investments in the classical model is the interest rate; which measures the cost of borrowing funds to finance investment. The higher the interest rate, the fewer investment projects that will be profitable, and hence the quantity of investment goods demanded will fall.

Since investment depends on the interest rate, we define the investment function such that it follows this intuition.\\
$I=I(r)$\\
Where:

\begin{itemize}
  \item I: Investment
  \item r: Real interest rate
\end{itemize}

Key characteristics of the investment function:

\begin{enumerate}
  \item Inverse relationship with the interest rate: As the interest rate increases, fewer investment projects are profitable, leading to a decrease in investment.
  \item Downward sloping: The investment demand curve slopes downward, reflecting the negative relationship between investment and the interest rate. That is, $\frac{\partial I}{\partial r}<0$\\
\includegraphics[max width=\textwidth, center]{2025_01_09_19fdeaed3c051bee82b7g-4}
\end{enumerate}

Mankiw, Macroeconomics, 10e, e 2019 Worth Publishers

If you look in the business section of a newspaper or on a financial website, you will find many different interest rates reported. By contrast, throughout this course, we talk about "the" interest rate, as if there were only one interest rate in the economy. The only distinction we make is between the nominal interest rate (which is not corrected for inflation) and the real interest rate (which is corrected for inflation). Almost all of the interest rates reported by financial news organizations are nominal.

Why are there so many interest rates? The various interest rates differ in three ways:

\begin{itemize}
  \item Term. Some loans in the economy are for short periods of time, even as short as overnight. Other loans are for thirty years or even longer. The interest rate on a loan depends on its term. Long-term interest rates are usually, but not always, higher than short-term interest rates.
  \item Credit risk. In deciding whether to make a loan, a lender must take into account the probability that the borrower will repay. The law allows borrowers to default on their loans by declaring bankruptcy. The higher the perceived probability of default, the higher the interest rate. Because the government has the lowest credit risk, government bonds tend to pay a low interest rate. At the other extreme, financially shaky corporations can raise funds only by issuing junk bonds, which pay a high interest rate to compensate for the high risk of default.
  \item Tax treatment. The interest on different types of bonds is taxed differently. Most important, when state and local governments issue bonds, called municipal bonds, the\\
holders of the bonds do not pay federal income tax on the interest income. Because of this tax advantage, municipal bonds pay a lower interest rate.
\end{itemize}

Although there are many different interest rates in the economy, macroeconomists often ignore these distinctions because the various interest rates tend to rise and fall together. For many purposes, we will not go far wrong by assuming there is only one interest rate. \textit{An interesting research question or project could be to study to what degree this assumption holds today.}

\section*{Limitations of the Classical Investment Function:}
\begin{enumerate}
  \item It assumes that firms have perfect information about future returns on investment projects, which is rarely the case in reality.
  \item It doesn't explicitly account for uncertainty and risk, which can significantly influence investment decisions.
  \item The model doesn't capture the potential for "animal spirits" (a term later popularized by Keynes) or irrational exuberance in driving investment booms and busts.
\end{enumerate}

Let's consider a simple linear investment function: $I=e-d r$\\
Where 'e' represents autonomous investment (investment that occurs regardless of the interest rate) and ' d ' is the sensitivity of investment to changes in the interest rate.

Suppose the investment function is given by I=1000-50r, where $r$ is the real interest rate expressed as a percentage.\\
a) Calculate the level of investment when the real interest rate is $5 \% . I=750$\\
b) If the interest rate increases to $\mathbf{7 \%}$, what is the change in investment? $\mathbf{I 2}=\mathbf{6 5 0} \boldsymbol{- >} \mathbf{- 1 0 0}$\\
c) What interest rate would result in zero investment according to this function? 20\%

\section*{Government}
The last component of our closed economy is the Government. As an economic entity, the government (federal or state) is able to create demand as well as purchase goods from existing industries. These can be in the form of government contracts, welfare, public employees, infrastructure projects, and so on. In the US, government purchases are estimated to be around $20 \%$ of GDP.

Importantly, we distinguish between two types of government purchases: spending and transfers. The key distinction is that spending is often in exchange for goods or services, while transfers are simply checks to another party. Thus, contracts and projects fall under spending (e.g., public sector wages, infrastructure projects, military expenses, ...) while welfare plans like\\
social security or medicare would fall under transfers. Since transfers do not directly relate to the production of goods and services in the economy, these are not included in our variable G.

Transfer payments do affect the demand for goods and services indirectly. Transfer payments are the opposite of taxes: they increase households' disposable income, just as taxes reduce disposable income. Thus, an increase in transfer payments financed by an increase in taxes leaves disposable income unchanged. We derive the following cases from this observation:

\begin{enumerate}
  \item If $G=T$, then the government has a balanced budget
  \item If $G<T$, then we have a budget deficit. Often funded by an increase in debt.
  \item If $\mathrm{G}>\mathrm{T}$, then we have a budget surplus which can be used to repay existing debt.
\end{enumerate}

For the purposes of this classical analysis, we assume that $G$ and $T$ are given. That is, these are exogenous variables. Denoted as $G=\bar{G}$ and $T=\bar{T}$. This assumption abstracts the political nuances that may give rise to the particular values of $G$ and $T$. Although very important to understand, considering the political economy as a whole requires more complex instruments for analyses. Some of these instruments include Agent based models, systems of differential equations, reinforcement learning, and other simulation based approaches that I hope to introduce briefly at the end of the semester.

We do, however, want to examine the impact of fiscal policy on the endogenous variables, which are determined within the model. The endogenous variables here are consumption, investment, and the interest rate.

Note that consumption and investment is up to the households or the firms of the economy. We cannot force people into making certain consumption or investment decisions. Thus the only degree of freedom in this model is the interest rate, the main policy instrument used by central banks. This will be our tool to bring, at least try to, the economy closer to equilibrium. Changes in G can also be used as a policy instrument, though classical economists generally advocated for limited government intervention.

Classical economists, influenced by Adam Smith's "invisible hand" concept, generally favored a limited role for government in the economy. They argued for:

\begin{itemize}
  \item Protecting property rights
  \item Providing public goods (e.g., national defense, basic infrastructure)
  \item Maintaining competitive markets
\end{itemize}

However, they were skeptical of large-scale government intervention, believing that markets would self-regulate in the long run.

Limitations of the Classical Approach to Government Expenditure:

\begin{enumerate}
  \item Assumes perfect information and rational expectations
  \item Neglects potential stimulative effects of government spending during recessions
  \item Doesn't account for potential productivity-enhancing effects of certain types of government expenditure (e.g., education, research funding)
\end{enumerate}

The Great Depression challenged classical assumptions about government's role, leading to the rise of Keynesian economics, which advocated for more active government intervention to stabilize the economy.

There are two ways to think about the role of the interest rate in the economy. We can consider how the interest rate affects the supply and demand for goods or services. Or we can consider how the interest rate affects the supply and demand for loanable funds. As we will see, these two approaches are two sides of the same coin.

Let's summarize the model we have built so far.\\
$Y=C+I+G$\\
$Y=F(K, L)=>\bar{Y}=F(\bar{K}, \bar{L})$, if we take factors of production as exogenous (since these are chosen by the firms and not by the modelers)\\
$C=C(\bar{Y}-\bar{T})$\\
$I=I(r)$\\
$G=\bar{G}$\\
$T=\bar{T}$

The demand for the economy's output comes from consumption, investment, and government purchases. Consumption depends on disposable income, investment depends on the real interest rate, and government purchases and taxes are the exogenous variables set by fiscal policymakers.

Putting everything together we have\\
$F(\bar{K}, \bar{L})=C(\bar{Y}-\bar{T})+I(r)+\bar{G}$\\
This equation states that the supply of output equals its demand, which is the sum of consumption, investment, and government purchases. It makes it extra clear that the only variable we get to manipulate is the interest rate $r$, everything else is already determined outside of the model (i.e, exogenously). Adjusting the interest rate will help us equilibrate supply and demand by increasing or decreasing investment. If the interest rate is too high, then investment is too low, and the demand for output falls short of the supply. If the interest rate is too low, then investment is too high, and the demand exceeds the supply. At the equilibrium interest rate, the demand for goods and services equals the supply.

Because the interest rate is the cost of borrowing and the return to lending in financial markets, we can better understand the role of the interest rate in the economy by thinking about the financial markets. Enter the loanable funds market.


\end{document}